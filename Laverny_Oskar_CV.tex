% LTeX: enabled=false
\documentclass[a4paper,11pt]{article}

\usepackage[english]{babel} 
\usepackage[T1]{fontenc}
\usepackage[utf8]{inputenc}
\usepackage[local,mark,raisemark=-0.05\paperheight]{gitinfo2}
\usepackage{booktabs}
\usepackage{multicol}
\usepackage{float}
\usepackage{lmodern}
\usepackage{setspace}
\usepackage{amsfonts}
\usepackage{ifthen}
\usepackage{parskip}


%%%%%%%%% Chnage git watermark 
% Default one : \renewcommand{\gitMark}{\gitBranch\,@\,\gitAbbrevHash{} \textbullet{} Release:\gitReln{} (\gitAuthorDate)\\ Head tags: \gitTags}
\renewcommand{\gitMark}{\gitBranch\,@\,\gitAbbrevHash{} \textbullet{} Date: \gitAuthorDate}
\renewcommand{\gitMarkPref}{\href{https://github.com/lrnv/cv}{[git repo]}}


%%%%%%%%% Put the number of pages on the right.
\usepackage{lastpage}
\makeatletter
\renewcommand{\@oddfoot}{\hfill \thepage/\pageref*{LastPage}}
\makeatother

%%%%%%%%% Hyperref setup
\usepackage{hyperref}
\definecolor{LightBlue}{RGB}{31,120,180}
\hypersetup{colorlinks,citecolor=LightBlue, filecolor=LightBlue,linkcolor=black,urlcolor=LightBlue}

%%%%%%%%% bibliography setup. 
\usepackage[backend=bibtex,style=ieee,sorting=ydnt,defernumbers=true,url=false]{biblatex}
\addbibresource{publi.bib}
\AtEveryBibitem{\ifentrytype{unpublished}{\clearname{author}}}
\DeclareFieldFormat[unpublished]{title}{#1} 



% Definition of the titles and stuff. 
%#########
\usepackage{titlesec}
\usepackage{vmargin}            % to redefine margins.
\setmarginsrb{2cm}{2cm}{2cm}{1cm}{0cm}{0cm}{0cm}{1cm} % Marge gauche, haute, droite, basse; espace entre la marge et le texte ?
  % gauche, en  haut, ? droite, en bas



% Pour laisser de l'espace entre les lignes du tableau
\newcommand{\HRule}[2]{{\centering\rule{#1}{#2}}}
\setlength{\arrayrulewidth}{0.4pt}

% Definition des sections 
\titleformat{\section}[block]{}{}{1em}
{
    \vspace{-1.3cm}
    \begin{flushleft}
    \begin{minipage}{\linewidth}
    \HRule{\linewidth}{0.2mm}\\[5pt]
    \centering
    \bf \thesection\quad
}
[
    \HRule{\linewidth}{0.2mm}
    \end{minipage}
    \end{flushleft}
]

\newcommand{\tabcv}[2]{
\begin{minipage}[t]{0.12\linewidth}
\textbf{\footnotesize #1}
\end{minipage}\hfill
\begin{minipage}[t]{0.85\linewidth}
#2
\end{minipage}
\vspace{1em}
}

\renewcommand{\tabcolsep}{0.1cm}
\renewcommand{\arraystretch}{1}

\definecolor{color0}{rgb}{0,0,0}% black
\definecolor{color1}{rgb}{0.22,0.45,0.70}% light blue
\definecolor{color2}{rgb}{0.45,0.45,0.45}% dark grey

\newcommand*{\namefont}{\fontsize{28}{30}\mdseries\upshape}
\newcommand*{\titlefont}{\LARGE\mdseries\slshape}
\newcommand*{\websitefont}{\mdseries\slshape}
\newcommand*{\addressfont}{\small\mdseries\slshape}
\newcommand*{\quotefont}{\large\slshape}

\newcommand*{\namestyle}[1]{{\namefont\textcolor{color0}{#1}}}
\newcommand*{\titlestyle}[1]{{\titlefont\textcolor{color2}{#1}}}
\newcommand*{\addressstyle}[1]{{\addressfont\textcolor{color2}{#1}}}

%###################
% LTeX: enabled=true
% lTeX: language=en
\begin{document}

\begin{multicols}{2}\raggedright
\namestyle{Oskar Laverny} \\
\vspace{0.5cm}
\titlestyle{Maître de Conférence}\\
\vspace{0.2cm}
\titlestyle{Actuaire Qualifié IA}
\columnbreak

\raggedleft
\addressstyle{
Aix-Marseille University\\
Marseille, France\\
\url{oskar.laverny@univ-amu.fr}\\
tèl : (+33) 777 838 406\\
\url{https://actuarial.science}\\
Build date: \today
}
\end{multicols}
\vspace{0.2cm}
\section{Current position}\label{subsec:act_pro_actuelle}

\begin{flushleft}
\tabcv{Since 2023}{
\textbf{Maître de Conférence (Assistant professor)},\\ \textit{UMR 1252 SESSTIM, Aix-Marseille Université}, \textit{Marseille, France}\\[0.5em]
{\footnotesize
\textbf{$\bullet$ Research:} My researches deal with high-dimensionnal dependence structures from both thoeretical and applied points of view: estimation, tests, sampling, etc. In particular: 
\begin{itemize}
  \item High dimensional statistics: sparse dependence structures, copulas \& factor models.
  \item Actuarial sciences: survival analysis, Non-life reserving, Dependence modeling \& more.
\end{itemize}
\textbf{$\bullet$ Teaching:} Statistics and computer sciences at SESSTIM, in M2 Artificial Intelligence for Public Health (AI4PH)  \& M2 Quatitative methods for health research (MQERS), in french and english:
\begin{itemize}
  \item Multivariate exploratory methods: PCA, ICA, PLS regression, ANOVA.
  \item Regression: Linear models, Logistic models, GLM, Mixed effects models.
  \item Machine learnings: K-means, SVM, Cart, Random forest, Neural nets. 
  \item Survival analysis: Cox-PH, NA, Kaplan-Meier, Mixed effects.
\end{itemize}
}
}

\section{Academic training}\label{subsec:formation_academique}

\tabcv{2019 - 2022}{
\textbf{PhD in statistics},\\ \textit{Institut Camille Jordan, Université Claude Bernard Lyon 1}, \textit{Lyon, France}, \\ \textit{SCOR SE, Paris, France}, \textit{}\\[0.5em]
%\hbox{\phantom{\ \ \ \ }}\begin{minipage}{0.97\linewidth}
{\footnotesize \textbf{$\bullet\ $Thesis title:} {Dependences structures and risk aggregation}\\
\textbf{$\bullet\ $Advisors:} Esterina Masiello, Véronique Maume-Deschamps and Didier Rullière.\\
\textbf{$\bullet\ $Keywords:} {Dependence structures, Thorin measures, Laguerre basis, copulas, non-parametric estimation, high-dimensional statistics}\\
\textbf{$\bullet\ $Earing date:} May the 30th, 2022\\
\textbf{$\bullet$ Funding:} Convention industrielle de formation par la recherche (CIFRE) between SCOR and University Lyon 1.\\
\textbf{$\bullet\ $Jury :} \\[0.25em]
\begin{tabular}{llll}
{Pr.} & {Anne-Laure Fougères}       & {Université Lyon 1} & {Présidente} \\
{Pr.} & {Fabrizio Durante}          & {Università del Salento} & {Rapporteur} \\
{Pr.} & {Edward Furman}             & {York University} & {Rapporteur} \\
{Pr.} & {Fabienne Comte}            & {Université Paris Descartes} & {Examinatrice} \\
{Pr.} & {Olivier Lopez}             & {Sorbonne Université} & {Examinateur} \\
{Dr.} & {Esterina Masiello}         & {Université Lyon 1} & {Directrice de thèse} \\
{Pr.} & {Véronique Maume-Deschamps} & {Université Lyon 1} & {Directrice de thèse} \\
% {Pr.} & {Didier Rullière}           & {Mines Saint-Étienne} & {Directeur de thèse} \\
% {Dr.} & {Alessandro Ferriero}       & {SCOR Zurich} & {Encadrant} \\
% {Mme.} & {Ecaterina Nisipasu}        & {SCOR Paris} & {Encadrante} \\
\end{tabular}}
}

\tabcv{2015 - 2018}{
\textbf{Diplôme d'actuaire}, \textit{ISFA}, \textit{Lyon}\\[0.5em]
{\footnotesize \textbf{$\bullet$ Actuarial thesis:} Reserving and reserve risk in (french) builder's insurence\\
\footnotesize \textbf{$\bullet$ Semester abroad:} at University of Waterloo, Ontario, Canada\\
\footnotesize \textbf{$\bullet$ Qualification:} Fully qualified actuary of the French Actuarial Institute since 2023. 
}
}

\tabcv{2012 - 2015}{
\textbf{Bachelor in Mathematics}, \textit{Université de Strasbourg}, \textit{Strasbourg}\\[0.5em]
}

\clearpage
\section{Past positions}

\tabcv{2022-2023}{
\textbf{Post-doctoral researcher}, \\ \textit{Institut de statistiques, biostatistiques et actuariat}, \textit{Louvain-La-Neuve, Belgique}\\[0.5em]
{\footnotesize
\textbf{$\bullet$ Research subject:} Penalised semi-parametric density estimation.\\
\textbf{$\bullet$ Funding:} FNRS grant for the ARC "Imperfect Data:
From Mathematical foundations to Applications in Life sciences" (IMAL).\\
\textbf{$\bullet$ Coauthor:} Philippe Lambert.
}
}


\tabcv{Summer 2022}{
\textbf{Post-doctoral researcher}, \textit{York University}, \textit{Toronto, Canada}\\[0.5em]
{\footnotesize
% \textbf{$\bullet$ Encadrants:} Edward Furman \& Ida Ferrara\\
\textbf{$\bullet$ Research subject:} Vulnerability theory and application to survey data around COVID-19. \\
\textbf{$\bullet$ Length:} Three months\\
\textbf{$\bullet$ Funding:} Postodctoral grant from the \textit{Fields Institute}\\
\textbf{$\bullet$ Coauthors:} Edward Furman \& Ida Ferara.
}
}


\tabcv{2019-2022}{
\textbf{PhD}, \textit{ICJ, Univ Lyon 1 \& SCOR SE}, \textit{Lyon \& Paris}\\[0.5em]
{\footnotesize
\textbf{$\bullet$ Research subjects:} Contributions of patchwork copulas and high dimensional generalized gamma convolutions to the modeling and estimations of dependence structures. See Section~\ref{subsec:formation_academique} for details.
}
}

\tabcv{2021-2022}{
\textbf{Lecturer}, \textit{Département d'économie, École normale supérieure}, \textit{Lyon}\\[0.5em]
{\footnotesize
\textbf{$\bullet$ Head of teaching dep.:} Sophie Hatte\\
\textbf{$\bullet$ Lecture topic:} Statistical inference\\
}
}

\tabcv{2020-2022}{
\textbf{Teaching assistant}, \textit{Département de mathématiques, Université Lyon 1}, \textit{Lyon}\\[0.5em]
{\footnotesize
\textbf{$\bullet$ Lecturers:} Yoann Dabrowski, Éric Delaygue\\
\textbf{$\bullet$ Subjects:} Probabilités et statistiques\\
}
}

\tabcv{2017-2018}{
\textbf{Alternant -- Chargé d'étude actuarielles}, \textit{L'Auxiliaire}, \textit{Lyon}\\[0.5em]
{\footnotesize
\textbf{$\bullet$ Supervisor:} Maxime Lenfant\\
\textbf{$\bullet$ Research subject:} Reserving and reserve risk in (french) builders insurence.\\
\textbf{$\bullet$ Other activities:} Implementation of reserving and risk analysis tools.\\
}
}


\tabcv{Summer 2016}{
\textbf{Intern}, \textit{Diacrisis}, \textit{Paris}\\[0.5em]
{\footnotesize
\textbf{$\bullet$ Supervisor:} Olivier Beruyer\\
\textbf{$\bullet$ Subject:} Implementation of economic and financial analysis. \\
}
}

\end{flushleft}

% \clearpage
\section{Languages and programming languages}
\tabcv{Languages}{
\textbf{French} native\\ 
\textbf{English} fluent\\ 
A few notions of \textbf{German} and \textbf{Esperanto}.\\
}
\tabcv{Programming languages}{
C++, Julia, R, Python, \LaTeX, git, bash, VBA, HTML/CSS/JS/PHP/SQL.\\
}


\section{Research outputs}
{
\renewcommand{\refname}{\vspace{-2.5em}}

\nocite{*}
\begin{itemize}

\item \textbf{Thesis manuscript}
\newrefcontext[labelprefix=M]
\printbibliography[keyword={thesis},heading=none]
% \printbibliography[prefixnumbers={M},keyword={thesis}]
\vspace{2em}


\item \textbf{Currently writting}
\newrefcontext[labelprefix=W]
\printbibliography[keyword={work_in_progress},heading=none]
% \printbibliography[prefixnumbers={E},keyword={work_in_progress}]
\vspace{2em}

\item \textbf{Submitted articles}
\newrefcontext[labelprefix=S]
\printbibliography[keyword={preprint},heading=none]
% \printbibliography[prefixnumbers={PP},keyword={preprints}]
\vspace{2em}

\item \textbf{Published articles}
\newrefcontext[labelprefix=P]
\printbibliography[keyword={article},heading=none]
% \printbibliography[prefixnumbers={P},keyword={articles}]
\vspace{2em}

\item \textbf{Non-peer-reviewed articles (vulgarization)}
\newrefcontext[labelprefix=V]
\printbibliography[keyword={non_peer_reviewed},heading=none]
% \printbibliography[prefixnumbers={V},keyword={non_peer_reviewed}]
\vspace{2em}

\item \textbf{Published softwares}
\newrefcontext[labelprefix=L]
\printbibliography[keyword={software},heading=none]
% \printbibliography[prefixnumbers={L},keyword={softwares}]
\vspace{2em}
% \clearpage
\item \textbf{Invited talks}
\newrefcontext[labelprefix=I]
\printbibliography[keyword={invited_talk},heading=none]
% \printbibliography[prefixnumbers={CI},keyword={invited_talks}]
\vspace{2em}

\item \textbf{Contributed talks}
\newrefcontext[labelprefix=C]
\printbibliography[keyword={contributed_talk},heading=none]
% \printbibliography[prefixnumbers={CC},keyword={contributed_talks}]
\vspace{2em}

\end{itemize}
}

\end{document}
