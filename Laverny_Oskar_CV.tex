%%%%%%%%%%%%%%%%%%%%%%%%%%%%%%%%%%%%%%%%%
% "ModernCV" CV and Cover Letter
% LaTeX Template
% Version 1.3 (29/10/16)
%
% This template has been downloaded from:
% http://www.LaTeXTemplates.com
%
% Original author:
% Xavier Danaux (xdanaux@gmail.com) with modifications by:
% Vel (vel@latextemplates.com)
%
% License:
% CC BY-NC-SA 3.0 (http://creativecommons.org/licenses/by-nc-sa/3.0/)
%
% Important note:
% This template requires the moderncv.cls and .sty files to be in the same 
% directory as this .tex file. These files provide the resume style and themes 
% used for structuring the document.
%
%%%%%%%%%%%%%%%%%%%%%%%%%%%%%%%%%%%%%%%%%
%----------------------------------------------------------------------------------------
%	PACKAGES AND OTHER DOCUMENT CONFIGURATIONS
%----------------------------------------------------------------------------------------
\documentclass[11pt,a4paper,sans]{moderncv} % Font sizes: 10, 11, or 12; paper sizes: a4paper, letterpaper, a5paper, legalpaper, executivepaper or landscape; font families: sans or roman
\moderncvstyle{classic} % CV theme - options include: 'casual' (default), 'classic', 'oldstyle' and 'banking'
\moderncvcolor{blue} % CV color - options include: 'blue' (default), 'orange', 'green', 'red', 'purple', 'grey' and 'black'
\usepackage[local,mark]{gitinfo2}
\usepackage[scale=0.90]{geometry} % Reduce document margins
\usepackage{multibib}
\newcites{%
articles,%
contributedtalks,%
invitedtalks,%
packages,%
preprints,%
toappear,%
wip%
}{%
{Name 1},%
{Name 2},%
{Name 3},%
{Name 4},%
{Name 5},%
{Name 6},%
{Name 7}}

\newcommand{\ENarticles}{
\subsection{Published articles (3)}
\nocitearticles{*}
\bibliographystylearticles{plainurlrev}
\bibliographyarticles{bib/articles}
}

\newcommand{\ENcontributedtalks}{
\subsection{Contributed talks (7)}
\nocitecontributedtalks{*}
\bibliographystylecontributedtalks{plainurlrev}
\bibliographycontributedtalks{bib/contributedtalks}
}

\newcommand{\ENinvitedtalks}{
\subsection{Invited talks (3)}
\nociteinvitedtalks{*}
\bibliographystyleinvitedtalks{plainurlrev}
\bibliographyinvitedtalks{bib/invitedtalks}
}

\newcommand{\ENpackages}{
\subsection{Softwares (2)}
\nocitepackages{*}
\bibliographystylepackages{plainurlrev}
\bibliographypackages{bib/packages}
}

\newcommand{\ENpreprints}{}

\newcommand{\ENtoappear}{}

\newcommand{\ENwip}{
\subsection{Work in progress (1)}
\nocitewip{*}
\bibliographystylewip{plainurlrev}
\bibliographywip{bib/wip}
}


%\setlength{\hintscolumnwidth}{3cm} % Uncomment to change the width of the dates column
%\setlength{\makecvtitlenamewidth}{10cm} % For the 'classic' style, uncomment to adjust the width of the space allocated to your name


\newcommand{\icj}{\href{http://math.univ-lyon1.fr/}{Institut Camille Jordan}}
\newcommand{\ucbl}{\href{https://www.univ-lyon1.fr/}{University Claude Bernard Lyon 1}}
\newcommand{\isba}{\href{https://uclouvain.be/en/research-institutes/lidam/isba}{Institut de statistiques, biostatistiques et actuariat}}
\newcommand{\ucl}{\href{https://uclouvain.be/}{Université Catholique de Louvain}}
\newcommand{\scor}{\href{https://www.scor.com}{SCOR SE}}
\newcommand{\ens}{\href{http://www.ens-lyon.fr/}{Ecole Normale Supérieure}}
\newcommand{\auxiliaire}{\href{http://www.auxiliaire.fr}{L'Auxilliaire}}
\newcommand{\diacrisis}{\href{https://www.les-crises.fr}{Diacrisis}}
\newcommand{\isfa}{\href{https://isfa.univ-lyon1.fr/}{Institut de Sciences Financières et d'Assurances}}
\newcommand{\unistra}{\href{https://www.unistra.fr/}{Université de Strasbourg}}
\newcommand{\risc}{\href{https://riscyu.org/}{RISC}}
\newcommand{\uyork}{\href{https://www.yorku.ca/}{York University}}
\newcommand{\sesstim}{\href{https://sesstim.univ-amu.fr/}{SESSTIM}}
\newcommand{\amu}{\href{https://univ-amu.fr}{Aix-Marseille Université}}

%----------------------------------------------------------------------------------------
%	NAME AND CONTACT INFORMATION SECTION
%----------------------------------------------------------------------------------------
\firstname{Oskar} % Your first name
\familyname{Laverny} % Your last name
% All information in this block is optional, comment out any lines you don't need
\title{Maitre de conférence\\ \newline Actuary (fully qualified)}
\address{\amu{}}{Marseille, France}
% \mobile{(+33) 777 838 406}
% \phone{(000) 111 1112}
% \fax{(000) 111 1113}
\email{oskar.laverny@univ-amu.fr}
\homepage{www.actuarial.science}{www.actuarial.science} % The first argument is the url for the clickable link, the second argument is the url displayed in the template - this allows special characters to be displayed such as the tilde in this example
\extrainfo{Build date: \today}
% \photo[120pt][0pt]{pictures/photo} % The first bracket is the picture height, the second is the thickness of the frame around the picture (0pt for no frame)
\quote{"What would be the dependence structure between quality of code and quantity of coffee ?"}
%----------------------------------------------------------------------------------------


\begin{document}
%%%%%%%%%%%%%%%%%%%%%%%%%%%%%%%%%%%%%%%%%%%%%%%%%%%%%%%%%%%%%%%%%%%%%%%%%%%%%%%%%%%%%%%%%%%%%%%%%%%%%
% Cover letter; 

% To remove the cover letter, comment out this entire block: 

% \clearpage

% \recipient{HR Department}{Corporation\\123 Pleasant Lane\\12345 City, State} % Letter recipient
% \date{\today} % Letter date
% \opening{Dear Sir or Madam,} % Opening greeting
% \closing{Sincerely yours,} % Closing phrase
% \enclosure[Attached]{curriculum vit\ae{}} % List of enclosed documents

% \makelettertitle % Print letter title

% \lipsum[1-2] % Dummy text
% \lipsum[4] % Dummy text

% \makeletterclosing % Print letter signature

% \newpage
%%%%%%%%%%%%%%%%%%%%%%%%%%%%%%%%%%%%%%%%%%%%%%%%%%%%%%%%%%%%%%%%%%%%%%%%%%%%%%%%%%%%%%%%%%%%%%%%%%%%%
\makecvtitle % Print the CV title


%%%%%%%%%%%%%%%%%%%%%%%%%%%%%%%%%%%%%%%%%%%%%%%%%%%%%%%%%%%%%%%%%%%%%%%%%%%%%%%%%%%%%%%%%%%%%%%%%%%%%



\section{Employment}
\cventry{Since 2023}{Maître de conférence\textit{(Assistant professor)}}{\sesstim{}, \amu{}}{Marseille, France}{}{}
\cventry{2022--2023}{Post-Doctoral researcher}{\isba{}, \ucl{}}{Louvain-la-Neuve, Belgium}{}{}
\cventry{Summer 2022}{Post-Doctoral researcher}{\risc{}, \uyork{}}{Toronto, Canada}{}{}
\cventry{2019--2022}{PhD Student}{\icj{}, \ucbl{} and \scor{}}{Lyon \& Paris, France}{}{}
\cventry{2021--2022}{Lecturer}{\ens{}}{Lyon}{}{}
\cventry{2020--2022}{Teaching assistant}{\icj{}, \ucbl{}}{Lyon}{}{}
\cventry{2017--2018}{Actuarial Analyst}{\auxiliaire{}}{Lyon}{}{}
\cventry{2016}{Statistical Intern}{\diacrisis{}}{Paris}{}{}

%%%%%%%%%%%%%%%%%%%%%%%%%%%%%%%%%%%%%%%%%%%%%%%%%%%%%%%%%%%%%%%%%%%%%%%%%%%%%%%%%%%%%%%%%%%%%%%%%%%%%
\section{Education}
\cventry{2019--2022}{PhD In Statistics}{\icj{}, \ucbl{} and \scor{}}{}{}{\href{https://www.actuarial.science/files/thesis.pdf}{Thesis supervised by V. Maume-Deschamps, E. Masiello and D. Rullière, under a CIFRE grant in partnership with SCOR SE. Title: "Dependence structures and risk aggregation in high dimensions". \emph{Defended on Mai 30, 2022}.}}
\cventry{2018--2019}{Master 2 research in probability}{\ens{}}{}{Auditing}{}
\cventry{2015--2018}{Bachelor, Master and DU in Actuarial sciences}{\isfa{}, \ucbl{}, with one semester at the University of Waterloo, Ontario, Canada}{}{Mathematics applied to insurance and finance}{\href{http://www.ressources-actuarielles.net/C12574E200674F5B/0/2B139F22F51AF75FC1258306006E82F2}{Master Thesis: Stochastic reserving in decennial insurance through generalized linear models}}
\cventry{2012--2015}{Bachelor in Mathematics}{\unistra{}}{}{}{}

%%%%%%%%%%%%%%%%%%%%%%%%%%%%%%%%%%%%%%%%%%%%%%%%%%%%%%%%%%%%%%%%%%%%%%%%%%%%%%%%%%%%%%%%%%%%%%%%%%%%%%
\section{Research Activities}
% One \ENxxxx per bib/xxx.bib file. everything else is automatic. 
\ENwip
\ENpreprints
\ENtoappear
\ENarticles
\ENnonpeerreview
\ENmanuscrit
\ENsoftwares
\ENinvitedtalks
\ENcontributedtalks

%%%%%%%%%%%%%%%%%%%%%%%%%%%%%%%%%%%%%%%%%%%%%%%%%%%%%%%%%%%%%%%%%%%%%%%%%%%%%%%%%%%%%%%%%%%%%%%%%%%%%
\section{Teaching Activities}
\cventry{2023+}{Maitre de conférence}{SESSTIM, AMU}{Marseille}{}{Statistics and computer sciences, for 2nd year master's students in \href{https://sesstim.univ-amu.fr/fr/master-ai4ph}{Artificial Intelligence for Public Health} and \href{https://sesstim.univ-amu.fr/fr/master-mqers}{Méthodes Quantitatives et Econométriques pour la Recherche en Santé}}
\cventry{Fall 2021}{Lecturer}{Eco. Dep. ENS Lyon}{Lyon}{}{Statistical Inference, 36h lectures, 3rd year bachelor in economics.}
\cventry{Falls 2020 \& 2021}{Teaching Assistant}{Math. Dep. Univ Lyon 1}{Lyon}{}{Probability and Statistics, 2$\times$36h tutorials, 2nd year bachelor in computer sciences.}
\cventry{Fall 2020}{Teaching Assistant}{Math. Dep. Univ Lyon 1}{Lyon}{}{Probability and Statistics, 30h tutorials, 3rd year bachelor in mathematics.}




%%%%%%%%%%%%%%%%%%%%%%%%%%%%%%%%%%%%%%%%%%%%%%%%%%%%%%%%%%%%%%%%%%%%%%%%%%%%%%%%%%%%%%%%%%%%%%%%%%%%%
\section{Others}
\subsection{Languages}
\cvitemwithcomment{Native}{Français}{Mother tongue}
\cvitemwithcomment{Fluent}{English}{Written \& oral}
\cvitemwithcomment{Beginner}{Deutsch, Esperanto}{Basic words and sentence only}
\cvitemwithcomment{Fluent}{C++}{Written only}

%%%%%%%%%%%%%%%%%%%%%%%%%%%%%%%%%%%%%%%%%%%%%%%%%%%%%%%%%%%%%%%%%%%%%%%%%%%%%%%%%%%%%%%%%%%%%%%%%%%%%
\subsection{Computer skills}
\cvitem{Day to Day}{Julia, C++, R, Python, \LaTeX, git}
\cvitem{Casual}{Sh, VBA, HTML/CSS/JS/PHP/SQL}

%%%%%%%%%%%%%%%%%%%%%%%%%%%%%%%%%%%%%%%%%%%%%%%%%%%%%%%%%%%%%%%%%%%%%%%%%%%%%%%%%%%%%%%%%%%%%%%%%%%%%
\subsection{Interests}
\cvitem{}{Free software, GNU/Linux, Lutherie}

\end{document}