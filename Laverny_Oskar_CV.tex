% LTeX: enabled=false
\documentclass[a4paper,11pt]{article}

\usepackage[english]{babel} 
\usepackage[T1]{fontenc}
\usepackage[utf8]{inputenc}
\usepackage[local,mark,raisemark=-0.05\paperheight]{gitinfo2}
\usepackage{lmodern}
\usepackage{multicol}
\usepackage{amsfonts}
\usepackage{ifthen}
\usepackage{parskip}

%%%%%%%%% Change git watermark 
% Default one : \renewcommand{\gitMark}{\gitBranch\,@\,\gitAbbrevHash{} \textbullet{} Release:\gitReln{} (\gitAuthorDate)\\ Head tags: \gitTags}
\renewcommand{\gitMark}{\gitBranch\,@\,\gitAbbrevHash{} \textbullet{} Date: \gitAuthorDate}
\renewcommand{\gitMarkPref}{\href{https://github.com/lrnv/cv}{[git repo]}}

%%%%%%%%% Put the number of pages on the right.
\usepackage{lastpage}
\makeatletter
\renewcommand{\@oddfoot}{\hfill \thepage/\pageref*{LastPage}}
\makeatother

%%%%%%%%% Hyperref setup
\usepackage{hyperref}
\definecolor{LightBlue}{RGB}{31,120,180}
\hypersetup{colorlinks,citecolor=LightBlue, filecolor=LightBlue,linkcolor=black,urlcolor=LightBlue}

%%%%%%%%% bibliography setup. 
\usepackage[backend=bibtex,style=ieee,sorting=ydnt,defernumbers=true,url=false]{biblatex}
\addbibresource{publi.bib}
\AtEveryBibitem{\ifentrytype{unpublished}{\clearname{author}}}
\DeclareFieldFormat[unpublished]{title}{#1} 
\renewcommand{\refname}{\vspace{-2.5em}}
% \renewcommand*{\bibfont}{\footnotesize}

% Definition of the titles and stuff. 
\usepackage{titlesec}
\usepackage{vmargin}            % to redefine margins.
\setmarginsrb{2cm}{2cm}{2cm}{1cm}{0cm}{0cm}{0cm}{1cm} % Marge gauche, haute, droite, basse; espace entre la marge et le texte ?
\setlength{\arrayrulewidth}{0.4pt}% Pour laisser de l'espace entre les lignes du tableau

% Definition des sections 
\titleformat{\section}[block]{}{}{1em}
{
    \vspace{-1.3cm}
    \begin{flushleft}
    \begin{minipage}{\linewidth}
    \centering\rule{\linewidth}{0.2mm}\\[5pt]
    \centering
    \bf \thesection\quad
}
[
    \centering\rule{\linewidth}{0.2mm}
    \end{minipage}
    \end{flushleft}
]

% Definition des subsections 
\titleformat{\subsection}[block]{}{}{1em}
{
   \kern-1.25em\bf\thesubsection\quad
    % \centering\bf\thesubsection\quad % to show the number and center.
}[]

% Definitions des subsectionbib.
\newcommand{\subsectionbib}[3]{
\subsection{#3}
\newrefcontext[labelprefix=#1]
\printbibliography[keyword={#2},heading=none]
\vspace{1em}
}

% Commande pour les entrées du CV: 
\newcommand{\tabcv}[2]{
\begin{minipage}[t]{0.12\linewidth}
\textbf{\footnotesize #1}
\end{minipage}\hfill
\begin{minipage}[t]{0.85\linewidth}
#2
\end{minipage}
\vspace{1em}
}

\renewcommand{\tabcolsep}{0.1cm}
\renewcommand{\arraystretch}{1}

\definecolor{DarkGrey}{rgb}{0.45,0.45,0.45}% dark grey
\newcommand*{\namestyle}[1]{{\fontsize{28}{30}\mdseries\upshape#1}}
\newcommand*{\titlestyle}[1]{{\LARGE\mdseries\slshape\textcolor{DarkGrey}{#1}}}
\newcommand*{\addressstyle}[1]{{\small\mdseries\slshape\textcolor{DarkGrey}{#1}}}

% \newcommand{\bull}[1]{\textbf{$\bullet\ $#1}}
\newcommand{\bull}[1]{\textbf{#1}}

%###################
% LTeX: enabled=true
% lTeX: language=en
\begin{document}

%%%%%%%% Header
\begin{multicols}{2}\raggedright
\namestyle{Oskar Laverny} \\
\vspace{0.5cm}
\titlestyle{Maître de Conférences}\\
\vspace{0.2cm}
\titlestyle{Actuaire Qualifié IA}
\columnbreak

\raggedleft
\addressstyle{
Aix-Marseille University\\
Marseille, France\\
\url{oskar.laverny@univ-amu.fr}\\
tèl : (+33) 777 838 406\\
\url{https://actuarial.science}\\
Build date: \today
}
\end{multicols}
\vspace{0.2cm}

\section{Current position}\label{subsec:act_pro_actuelle}

\tabcv{Since 2023}{
\textbf{Maître de Conférences (Assistant professor)},\\
\textit{UMR 1252 SESSTIM, Aix-Marseille Université}, \textit{Marseille, France} \\[0.5em]
{\footnotesize
  \bull{Research:} My researches deal with dependence structures from both thoeretical and applied points of view: estimation, tests, sampling, etc. In particular: 
  \begin{itemize}
    \item High dimensional statistics: sparse dependence structures, copulas \& factor models.
    \item Actuarial sciences: survival analysis, Non-life reserving, Dependence modeling \& more.
    \item Biostatistical applications: Relative survival analysis, cancer registries, net survival, etc.
  \end{itemize}
  \bull{Teaching:} Statistics and computer sciences at SESSTIM, in M2 Artificial Intelligence for Public Health (AI4PH)  \& M2 Quantitative methods for health research (MQERS), in French and English:
  \begin{itemize}
    \item Multivariate exploratory methods: PCA, ICA, PLS regression, ANOVA.
    \item Regression: Linear models, Logistic models, GLM, Mixed effects models.
    \item Machine learnings: K-means, SVM, Cart, Random forest, Neural nets. 
    \item Survival analysis: Cox-PH, NA, Kaplan-Meier, Mixed effects.
    \item Computer science: R, Python.
  \end{itemize}
}
}

\section{Academic training}\label{subsec:formation_academique}

\tabcv{2019 - 2022}{
\textbf{PhD in statistics},\\ 
\textit{Institut Camille Jordan, Université Claude Bernard Lyon 1}, \textit{Lyon, France}, \\
\textit{SCOR SE, Paris, France}, \\[0.5em]
{\footnotesize 
  \bull{Thesis title:} {Dependences structures and risk aggregation}\\
  \bull{Advisors:} Esterina Masiello, Véronique Maume-Deschamps and Didier Rullière.\\
  \bull{Keywords:} {Dependence structures, Thorin measures, Laguerre basis, copulas, non-parametric estimation, high-dimensional statistics}\\
  \bull{Earing date:} May the 30th, 2022\\
  \bull{Funding:} Convention industrielle de formation par la recherche (CIFRE), SCOR \& Univ. Lyon 1.\\
  \bull{Jury:} \\[0.25em]
  \begin{tabular}{llll}
    {Pr.} & {Anne-Laure Fougères}       & {Université Lyon 1} & {Présidente} \\
    {Pr.} & {Fabrizio Durante}          & {Università del Salento} & {Rapporteur} \\
    {Pr.} & {Edward Furman}             & {York University} & {Rapporteur} \\
    {Pr.} & {Fabienne Comte}            & {Université Paris Descartes} & {Examinatrice} \\
    {Pr.} & {Olivier Lopez}             & {Sorbonne Université} & {Examinateur} \\
    {Dr.} & {Esterina Masiello}         & {Université Lyon 1} & {Directrice de thèse} \\
    {Pr.} & {Véronique Maume-Deschamps} & {Université Lyon 1} & {Directrice de thèse} \\
    % {Pr.} & {Didier Rullière}           & {Mines Saint-Étienne} & {Directeur de thèse} \\
    % {Dr.} & {Alessandro Ferriero}       & {SCOR Zurich} & {Encadrant} \\
    % {Mme.} & {Ecaterina Nisipasu}       & {SCOR Paris} & {Encadrante} \\
  \end{tabular}
}
}

\tabcv{2015 - 2018}{
\textbf{Diplôme d'actuaire}, \textit{ISFA}, \textit{Lyon}\\[0.5em]
{\footnotesize 
  \bull{Actuarial thesis:} Reserving and reserve risk in (french) builder's insurence\\
  \bull{Semester abroad:} at University of Waterloo, Ontario, Canada\\
  \bull{Qualification:} Fully qualified actuary of the French Actuarial Institute since 2023. 
}
}

\tabcv{2012 - 2015}{
\textbf{Bachelor in Mathematics}, \textit{Université de Strasbourg}, \textit{Strasbourg}\\[0.5em]
}

\clearpage
\section{Past positions}

\tabcv{2022-2023}{
\textbf{Post-doctoral researcher}, \\
\textit{Institut de statistiques, biostatistiques et actuariat}, \textit{Louvain-La-Neuve, Belgique}\\[0.5em]
{\footnotesize
  \bull{Research subject:} Penalised semi-parametric density estimation.\\
  \bull{Funding:} FNRS grant for the ARC "Imperfect Data:
  From Mathematical foundations to Applications in Life sciences" (IMAL).\\
  \bull{Coauthor:} Philippe Lambert.
}
}


\tabcv{Summer 2022}{
\textbf{Post-doctoral researcher}, \textit{York University}, \textit{Toronto, Canada}\\[0.5em]
{\footnotesize
  \bull{Research subject:} Vulnerability theory and application to survey data around COVID-19. \\
  \bull{Length:} Three months\\
  \bull{Funding:} Postodctoral grant from the \textit{Fields Institute}\\
  \bull{Coauthors:} Edward Furman \& Ida Ferara.
}
}


\tabcv{2019-2022}{
\textbf{PhD}, \textit{ICJ, Univ Lyon 1 \& SCOR SE}, \textit{Lyon \& Paris}\\[0.5em]
{\footnotesize
  \bull{Research subjects:} Contributions of patchwork copulas and high dimensional generalized gamma convolutions to the modeling and estimations of dependence structures. See Section~\ref{subsec:formation_academique} for details.
}
}

\tabcv{2021-2022}{
\textbf{Lecturer}, \textit{Département d'économie, École normale supérieure}, \textit{Lyon}\\[0.5em]
{\footnotesize
  \bull{Head of teaching dep.:} Sophie Hatte\\
  \bull{Lecture topic:} Statistical inference, bachelor level.\\
}
}

\tabcv{2020-2022}{
\textbf{Teaching assistant}, \textit{Département de mathématiques, Université Lyon 1}, \textit{Lyon}\\[0.5em]
{\footnotesize
  \bull{Lecturers:} Yoann Dabrowski, Éric Delaygue\\
  \bull{Subjects:} Probabilities and statistics, bachelor level.\\
}
}

\tabcv{2017-2018}{
\textbf{Alternant -- Chargé d'étude actuarielles}, \textit{L'Auxiliaire}, \textit{Lyon}\\[0.5em]
{\footnotesize
  \bull{Supervisor:} Maxime Lenfant\\
  \bull{Research subject:} Reserving and reserve risk in (french) builders insurence.\\
  \bull{Other activities:} Implementation of reserving and risk analysis tools.\\
}
}


\tabcv{Summer 2016}{
\textbf{Intern}, \textit{Diacrisis}, \textit{Paris}\\[0.5em]
{\footnotesize
  \bull{Supervisor:} Olivier Beruyer\\
  \bull{Subject:} Implementation of economic and financial analysis. \\
}
}

\section{Languages and programming languages}
\tabcv{Languages}{
  \textbf{French} native\\ 
  \textbf{English} fluent\\ 
  A few notions of \textbf{German} and \textbf{Esperanto}.\\
}
\tabcv{Programming languages}{C++, Julia, R, Python, \LaTeX, git, bash, VBA, HTML/CSS/JS/PHP/SQL.}

\section{Responsabilities, fundings and research stays}

\subsection{Editorial work}
\tabcv{Since 2023}{\textbf{Editor} for the {{Journal of Open Source Software}}.}

\subsection{Management}
\tabcv{Summer 2024}{\textbf{M2 Internship} of Rim Alahjal\\
{\footnotesize
Topic: Relative survival analysis in Julia. Currently doing a PhD, CNRS/INRIA, Grenoble
}}

\subsection{Research stays}
\tabcv{June 2024}{\textbf{Invited scholar}\\
{\footnotesize
Invited at University College London by Dr F. Javier Rubio (one week).
}}

\subsection{Event organization}
\tabcv{July 2024}{\textbf{Member of the organizing comitee of HEARSTAT2024}\\
{\footnotesize
4th Corsican Summer School on Modern Methods in Biostatistics and Epidemiology.
}}

\section{Research outputs}
\nocite{*}
% LTeX: enabled=false
\subsectionbib{W}{work_in_progress} {Currently writting}
\subsectionbib{S}{preprint}         {Submitted articles}
\subsectionbib{P}{article}          {Published articles}
\subsectionbib{V}{non_peer_reviewed}{Non-peer-reviewed articles (vulgarization)}
\subsectionbib{L}{software}         {Open source software packages}
\subsectionbib{M}{thesis}           {Thesis manuscript}
\subsectionbib{I}{invited_talk}     {Invited talks}
\subsectionbib{C}{contributed_talk} {Contributed talks}
\end{document}
