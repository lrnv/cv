% LTeX: enabled=false
\documentclass[a4paper,11pt]{article} % fonte 11 points, papier a4

\usepackage[french,english]{babel}   
\usepackage[T1]{fontenc}        
\usepackage{url}                
\usepackage[utf8]{inputenc}
\usepackage{booktabs}
\usepackage{float}
\usepackage{lmodern}
\usepackage{setspace}
\usepackage{textcomp}
\usepackage{amsfonts}
\usepackage[table,usenames,dvipsnames]{xcolor}
\usepackage{bm}
\usepackage{graphicx}
\usepackage{multirow}
\usepackage{tikz}
\usepackage{fancybox}
\usepackage{colortbl}
\usepackage[final]{pdfpages}
\usepackage{multicol}
\usepackage{enumitem}
\usepackage{ifthen}
\usepackage{parskip}

\definecolor{Paired-1}{RGB}{31,120,180}

\usepackage[backend=bibtex,style=ieee,sorting=ynt,defernumbers=true]{biblatex}

\DefineBibliographyStrings{english}{
  url=,
}
\addbibresource{publi.bib}
\DeclareFieldFormat[unpublished]{title}{#1} 
\usepackage[final]{pdfpages}
\usepackage{csquotes}

\usepackage[nomessages]{fp}

\newcommand{\myname}{\textbf{O. Laverny}}

\usepackage{hyperref}
\hypersetup{colorlinks,citecolor=Paired-1, filecolor=Paired-1,linkcolor=black,urlcolor=Paired-1}

% La page
%#########
\usepackage{titlesec}
\usepackage{vmargin}            % redefinir les marges
\setmarginsrb{2cm}{2cm}{2cm}{1cm}{0cm}{0cm}{0cm}{1cm}
    
% Marge gauche, haute, droite, basse; espace entre la marge et le texte ?
% gauche, en  haut, ? droite, en bas

%%Cleardoublepage
\makeatletter
\renewcommand{\cleardoublepage}{
\clearpage\thispagestyle{empty}
\if@twoside
\ifodd\c@page
\else
\hbox{}\newpage
\fi
\fi
}
\makeatother

% Pour laisser de l'espace entre les lignes du tableau
\newcommand{\HRule}[2]{{\centering\rule{#1}{#2}}}

\definecolor{lightlightblue}{rgb}{0.75,0.85,1}
\definecolor{lightlightgray}{rgb}{0.93,0.93,0.93}
\definecolor{lightlightgray2}{rgb}{0.8,0.8,0.8}
\definecolor{lightlightlightgray}{rgb}{0.98,0.98,0.98}


\setlength{\arrayrulewidth}{0.4pt}

\setcounter{tocdepth}{2}

\titleformat{\section}[block]{}{}{1em}
{
    \vspace{-1.3cm}
    \begin{flushleft}
    \begin{minipage}{\linewidth}
    \HRule{\linewidth}{0.2mm}\\[5pt]
    \centering
    \bf \thesection\quad
}
[
    \HRule{\linewidth}{0.2mm}
    \end{minipage}
    \end{flushleft}
]

% \titleformat{\section}[block]{\Large \sc}{\thesection}{1em}{\centering}

\newcommand{\tabcv}[2]{
\begin{minipage}[t]{0.12\linewidth}
\textbf{\footnotesize #1}
\end{minipage}\hfill
\begin{minipage}[t]{0.85\linewidth}
#2
\end{minipage}
\vspace{1em}
}

\renewcommand{\tabcolsep}{0.1cm}
\renewcommand{\arraystretch}{1}

\definecolor{color0}{rgb}{0,0,0}% black
\definecolor{color1}{rgb}{0.22,0.45,0.70}% light blue
\definecolor{color2}{rgb}{0.45,0.45,0.45}% dark grey

\newcommand*{\namefont}{\fontsize{28}{30}\mdseries\upshape}
\newcommand*{\titlefont}{\LARGE\mdseries\slshape}
\newcommand*{\websitefont}{\mdseries\slshape}
\newcommand*{\addressfont}{\small\mdseries\slshape}
\newcommand*{\quotefont}{\large\slshape}

\newcommand*{\namestyle}[1]{{\namefont\textcolor{color0}{#1}}}
\newcommand*{\titlestyle}[1]{{\titlefont\textcolor{color2}{#1}}}
\newcommand*{\addressstyle}[1]{{\addressfont\textcolor{color2}{#1}}}

%###################
% LTeX: enabled=true
% lTeX: language=fr
\begin{document}

\begin{multicols}{2}\raggedright
\namestyle{Oskar Laverny} \\
\vspace{0.5cm}
\titlestyle{Maître de Conférence}\\
\vspace{0.2cm}
\titlestyle{Actuaire Qualifié IA}
\columnbreak

\raggedleft
\addressstyle{
Aix-Marseille University\\
Marseille, France\\
e-mail : \url{oskar.laverny@univ-amu.com}\\
tèl : (+33) 777 838 406\\
28 ans\\
\url{https://actuarial.science}
}
\end{multicols}
\vspace{0.2cm}
\section{Activité professionnelle actuelle}\label{subsec:act_pro_actuelle}

\begin{flushleft}
\tabcv{Depuis 2023}{
\textbf{Maître de Conférence}, \\ \textit{Aix-Marseille Université, UMR 1252 SESSTIM}, \textit{Marseille, France}\\[0.5em]
{\footnotesize
\textbf{$\bullet$ Enseignement} : Statistiques et informatique, M2 Artificial Intelligence for Public Health (AI4PH) et Méthodes Quantitatives et Econométriques pour la Recherche en Santé (MQERS) du SESSTIM\\
\textbf{$\bullet$ Recherche} : Mes recherches tournent autour des structures de dépendances en grandes dimensions d'un point de vue statistique, théorique comme appliqué: estimation, tests, simulation...
}
}

\section{Formation académique}\label{subsec:formation_academique}

\tabcv{2019 - 2022}{
\textbf{Doctorat en statistiques}, \\ \textit{Université Claude Bernard Lyon 1}, \textit{Institut Camille Jordan}, \textit{Lyon, France}, \\ \textit{SCOR SE, Paris, France}, \textit{}\\[0.5em]
%\hbox{\phantom{\ \ \ \ }}\begin{minipage}{0.97\linewidth}
{\footnotesize \textbf{$\bullet\ $Sujet de thèse :} {Structures de dépendance et agrégation des risques}\\
\textbf{$\bullet\ $Directeurs de thèse:} Esterina Masiello, Véronique Maume-Deschamps et Didier Rullière.\\
\textbf{$\bullet\ $Mots clefs :} {Structures de dépendances, mesures de Thorin, base de Laguerre, copules, estimation non paramétrique, statistiques en grandes dimensions}\\
\textbf{$\bullet\ $Date de soutenance :} 30 mai 2022\\
\textbf{$\bullet$ Financement} : Convention industrielle de Formation par la recherche (CIFRE) entre la SCOR et l'Université Lyon 1.\\
\textbf{$\bullet\ $Jury :} \\[0.25em]
\begin{tabular}{llll}
{Pr.} & {Anne-Laure Fougères}       & {Université Lyon 1} & {Présidente} \\
{Pr.} & {Fabrizio Durante}          & {Università del Salento} & {Rapporteur} \\
{Pr.} & {Edward Furman}             & {York University} & {Rapporteur} \\
{Pr.} & {Fabienne Comte}            & {Université Paris Descartes} & {Examinatrice} \\
{Pr.} & {Olivier Lopez}             & {Sorbonne Université} & {Examinateur} \\
{Dr.} & {Esterina Masiello}         & {Université Lyon 1} & {Directrice de thèse} \\
{Pr.} & {Véronique Maume-Deschamps} & {Université Lyon 1} & {Directrice de thèse} \\
% {Pr.} & {Didier Rullière}           & {Mines Saint-Étienne} & {Directeur de thèse} \\
% {Dr.} & {Alessandro Ferriero}       & {SCOR Zurich} & {Encadrant} \\
% {Mme.} & {Ecaterina Nisipasu}        & {SCOR Paris} & {Encadrante} \\
\end{tabular}}
}

\tabcv{2015 - 2018}{
\textbf{Diplôme d'actuaire}, \textit{ISFA}, \textit{Lyon}\\[0.5em]
{\footnotesize \textbf{$\bullet$ Sujet de mémoire :} Provisionnement et risque de réserve en assurance construction\\
\footnotesize \textbf{$\bullet$ Semestre d'échange :} University of Waterloo, Ontario, Canada\\
\footnotesize \textbf{$\bullet$ Qualification :} Actuaire qualifié de l'IA depuis janvier 2023. 
}
}

\tabcv{2012 - 2015}{
\textbf{Licence de mathématiques}, \textit{Université de Strasbourg}, \textit{Strasbourg}\\[0.5em]
% {\footnotesize\textbf{$\bullet$ Sp?cialit? :} Mesures Physiques }
}

\clearpage
\section{Expériences professionnelles antérieures}


% \tabcv{Depuis 2023}{
% \textbf{Maître de Conférence}, \\ \textit{Aix-Marseille Université, UMR 1252 SESSTIM}, \textit{Marseille, France}\\[0.5em]
% {\footnotesize
% % \textbf{$\bullet$ Enseignements} : Statistiques et informatiques\\
% }
% }

\tabcv{2022-2023}{
\textbf{Chercheur Post Doctoral}, \\ \textit{Institut de statistiques, biostatistiques et actuariat}, \textit{Louvain-La-Neuve, Belgique}\\[0.5em]
{\footnotesize
% \textbf{$\bullet$ Encadrant} : Philippe Lambert\\
\textbf{$\bullet$ Thème de recherche} : \'Estimation pénalisée de densités semi-paramétriques.\\
\textbf{$\bullet$ Financement} : Bourse FNRS dans le cadre de l'ARC "Imperfect Data:
From Mathematical foundations to Applications in Life sciences" (IMAL).\\
}
}


\tabcv{\'Eté 2022}{
\textbf{Chercheur Post-Doctoral}, \textit{York University}, \textit{Toronto, Canada}\\[0.5em]
{\footnotesize
% \textbf{$\bullet$ Encadrants} : Edward Furman \& Ida Ferrara\\
\textbf{$\bullet$ Thème de recherche} : Formalisation d'une théorie de la vulnérabilité et application à des données de sondages relatives au Covid-19. \\
\textbf{$\bullet$ Durée} : Trois mois\\
\textbf{$\bullet$ Financement} : Bourse postdoctorale du \textit{Fields Institute}\\
}
}


\tabcv{2019-2022}{
\textbf{Doctorant}, \textit{ICJ, Univ Lyon 1 \& SCOR SE}, \textit{Lyon \& Paris}\\[0.5em]
{\footnotesize
\textbf{$\bullet$ Thèmes de recherche} : Apport des copules de type patchwork et des convolutions généralisées de lois Gammas en grande dimension à la modélisation et l'estimation des structures de dépendance. \\
\textbf{$\bullet$ Détails} : Voir Section~\ref{subsec:formation_academique}\\
}
}

\tabcv{2021-2022}{
\textbf{Chargé de cours}, \textit{Département d'économie, École normale supérieure}, \textit{Lyon}\\[0.5em]
{\footnotesize
\textbf{$\bullet$ Responsable de l'enseignement du département} : Sophie Hatte\\
\textbf{$\bullet$ Sujet} : Inférence statistique\\
}
}

\tabcv{2020-2022}{
\textbf{Chargé de TD}, \textit{Département de mathématiques, Université Lyon 1}, \textit{Lyon}\\[0.5em]
{\footnotesize
\textbf{$\bullet$ Responsables} : Yoann Dabrowski, Éric Delaygue\\
\textbf{$\bullet$ Sujet} : Probabilités et statistiques\\
}
}

\tabcv{2017-2018}{
\textbf{Alternant -- Chargé d'étude actuarielles}, \textit{L'Auxiliaire}, \textit{Lyon}\\[0.5em]
{\footnotesize
\textbf{$\bullet$ Encadrant} : Maxime Lenfant\\
\textbf{$\bullet$ Sujet de mémoire} : Provisionnement et risque de réserve en assurance construction\\
\textbf{$\bullet$ Activités annexes} : Automatisation des outils de provisionnement et d'analyse des risques\\
}
}


\tabcv{Mai-Août 2016}{
\textbf{Stagiaire}, \textit{Diacrisis}, \textit{Paris}\\[0.5em]
{\footnotesize
\textbf{$\bullet$ Encadrant} : Olivier Beruyer\\
\textbf{$\bullet$ Sujet} : Automatisation d'analyses économiques et financières\\
}
}

\end{flushleft}


\section{Langues et langages de programation}
\tabcv{Langues}{
\textbf{Français} natif\\ 
\textbf{Anglais} courant\\ 
Quelques notions d'\textbf{Allemand} et d'\textbf{Esperanto}.\\
}
\tabcv{Langages}{
C++, Julia, R, Python, \LaTeX, git, bash, VBA, HTML/CSS/JS/PHP/SQL.\\
}

\clearpage
\section{Publications et conférences}
{
\renewcommand{\refname}{\vspace{-2.5em}}

\nocite{*}
\begin{itemize}

\item \textbf{Manuscrit de thèse}
\newrefcontext[labelprefix=M]
\printbibliography[keyword={thesis},heading=none]
% \printbibliography[prefixnumbers={M},keyword={thesis}]
\vspace{2em}


\item \textbf{En cours d'écriture}
\newrefcontext[labelprefix=E]
\printbibliography[keyword={work_in_progress},heading=none]
% \printbibliography[prefixnumbers={E},keyword={work_in_progress}]
\vspace{2em}

\item \textbf{Articles soumis pour publications}
\newrefcontext[labelprefix=S]
\printbibliography[keyword={preprint},heading=none]
% \printbibliography[prefixnumbers={PP},keyword={preprints}]
\vspace{2em}

\item \textbf{Articles publiés}
\newrefcontext[labelprefix=A]
\printbibliography[keyword={article},heading=none]
% \printbibliography[prefixnumbers={P},keyword={articles}]
\vspace{2em}

\item \textbf{Articles de vulgarisation (sans revue par les pairs)}
\newrefcontext[labelprefix=V]
\printbibliography[keyword={non_peer_reviewed},heading=none]
% \printbibliography[prefixnumbers={V},keyword={non_peer_reviewed}]
\vspace{2em}

\item \textbf{Logiciels publiés}
\newrefcontext[labelprefix=L]
\printbibliography[keyword={software},heading=none]
% \printbibliography[prefixnumbers={L},keyword={softwares}]
\vspace{2em}
% \clearpage
\item \textbf{Conférences (invitations)}
\newrefcontext[labelprefix=I]
\printbibliography[keyword={invited_talk},heading=none]
% \printbibliography[prefixnumbers={CI},keyword={invited_talks}]
\vspace{2em}

\item \textbf{Conférences (contributions)}
\newrefcontext[labelprefix=C]
\printbibliography[keyword={contributed_talk},heading=none]
% \printbibliography[prefixnumbers={CC},keyword={contributed_talks}]
\vspace{2em}

\end{itemize}
}

\end{document}
